\documentclass[a4paper,12pt]{article}
\usepackage{cmap}
\usepackage[utf8]{inputenc}
\usepackage[warn]{mathtext}
\usepackage{epsf,amsmath,amsfonts,amssymb,amsbsy}
\usepackage[mathscr]{eucal}
\usepackage[english, russian]{babel}
\author{Макар Шевцов Б02-207}
\title{Отчёт о выполнении лабораторной работы 2.1.1}
\usepackage[left=2cm,right=2cm,top=2cm,bottom=2cm]{geometry}
\usepackage{graphicx}
\usepackage{tabularx}
\usepackage{wrapfig}
\usepackage{graphicx}
\usepackage{indentfirst}
\graphicspath{{pictures2.1.1/}}
\DeclareGraphicsExtensions{.pdf,.png,.jpg}
\usepackage{pgfplots}
\begin{document}
	\maketitle
	\begin{center}
		{\Large Измерение удельной теплоёмкости воздуха при постоянном давленини}
	\end{center}
	\paragraph*{Цель работы:} измерить повышение температуры воздуха в зависимости от мощности
	подводимого тепла и расхода при стационарном течении через трубу; исключив тепловые потери, по результатам измерений определить теплоёмкость воздуха при постоянном давлении.
	\paragraph*{В работе используются:} теплоизолированная стеклянная трубка; электронагреватель; источник питания постоянного тока; амперметр, вольтметр (цифровые мультиметры); термопара, подключенная к микровольтметру; компрессор; газовый счётчик;
	секундомер.
	\section{Теоретическое введение}
	
		 Теплоёмкость тела в некотором процессе определяется как их отношение: $$ C = \frac{\delta Q}{dT} \; (1).$$
		 
		 Необходимо, чтобы количество тепла, затрачиваемое на нагревание исследуемого тела, существенно превосходило тепло, расходуемое на нагревание самого калориметра, а также на потери тепла из установки.
		 
		 Для увеличения количества нагреваемого газа при неизменных размерах установки в нашей работе исследуемый газ (воздух) продувается через калориметр, внутри которого установлен нагреватель. При этом
		 измеряются мощность нагревателя, масса воздуха, протекающего в единицу
		 времени (расход), и приращение его температуры.
		 
		 \begin{figure}[h]
		 \center{\includegraphics{lab_2_1_1}}
		 \end{figure}
		 Рассмотрим газ, протекающий
		 стационарно слева направо через
		 трубу постоянного сечения, в которой установлен нагревательный элемент (см. рис. 1). Пусть за некоторое
		 время $dt$ через калориметр прошла
		 малая порция газа массой $dm=q dt$,
		 где $q$ [кг/с] — массовый расход газа в трубе. Если мощность нагрева равна $N$, мощность тепловых потерь на обмен с окружающей средой $N_{пот}$, то порция
		 получила тепло $\delta Q = (N-N_{пот})dt$. С другой стороны, по определению теплоёмкости (1): $\delta Q = c dm \Delta T$, где $\Delta T = T_{2}-T_{1}$ — приращение температуры	газа, и $c$ — удельная (на единицу массы) теплоёмкость газа в рассматриваемом процессе. При малых расходах газа и достаточно большом диаметре
		 трубы перепад давления на её концах мал, поэтому можно принять, что $P_{1} \approx P_{2} = P_{0}$, где $P_{0}$ — атмосферное давление. Следовательно, в условиях опыта
		 измеряется удельная теплоёмкость при постоянном давлении $c_{P}$. Таким образом, получаем $$c_{P} = \frac{N-N_{пот}}{q\Delta T} \; (2).$$
		 
	\subsection{Эксперементальная установка:}
	
	Схема установки изображена на рис. 2. Воздух, нагнетаемый компрессором, прокачивается через калориметр. Калориметр представляет собой стеклянную цилиндрическую трубку с двойными стенками, запаянными с торцов.
	
		\begin{figure}[h]
		\center{\includegraphics{lab_2_1_1_ust}}
		\end{figure}
	
		Нагреватель в виде намотанной на пенопласт нихромовой проволоки расположен внутри калориметра непосредственно в воздушном потоке. Нагрев проволоки производится от регулируемого источника постоянного тока (ИП).
		Напряжение $U$ на нагревателе и ток $I$ через него регистрируются цифровыми мультиметрами. Таким образом, мощность нагрева равна
		$$N= UI \; (3).$$ 
		Для измерения разности температур $\Delta T$ служит медно-константановая
		термопара. Один спай термопары расположен в струе воздуха, входящего в
		калориметр, и находится при комнатной температуре, а второй — в струе выходящего нагретого воздуха. Константановая проволока термопары расположена внутри калориметра, а медные проводники подключены к цифровому вольтметру. Возникающая в термопаре ЭДС $\varepsilon$ пропорциональна разности температур $\Delta T$ спаев: $$\varepsilon =\beta \Delta T \; (4),$$ где $\beta = 40.7 \frac{мкВ}{^\circ C}$ — чувствительность медно-константановой термопары в рабочем диапазоне температур (20–30 $^\circ C$ ). ЭДС регистрируется с помощью микровольтметра.
		
		Объём воздуха, прошедшего через калориметр, измеряется газовым счётчиком ГС. Для регулировки расхода служит кран К. Время $\Delta t$ прохождения
		некоторого объема $\Delta V$ воздуха измеряется секундомером. Объёмный расход равен $\frac{\Delta V}{\Delta t} $, массовый расход может быть найден как $$q = \rho_{0} \frac{\Delta V}{\Delta t} \; (5),$$ где $rho_{0}$ — плотность воздуха при комнатной температуре, которая в свою очередь может быть получена из уравнения Менделеева–Клапейрона: $\rho_{0}= \frac{\mu P_{0} }{R T_{0}},$ где $P_{0}$ — атмосферное давление, $T_{0}$ — комнатная температура (в Кельвинах), $\mu = 29,0 {г/моль}$ — средняя молярная масса (сухого) воздуха.
		
		Учитывая особенности устройства калориметра, следует ожидать, что мощность нагревателя расходуется не только на нагрев массы прокачиваемого воздуха, но и частично теряется за счет нагрева внутренних стенок термостата и рассеяния тепла через торцы термостата. Можно предположить, что при небольшом нагреве ($\Delta T \ll T_{0}$) мощность потерь тепла $N_{пот}$ прямо пропорциональна разности температур:$$ N_{пот} = \alpha \Delta T \; (6),$$ где $\alpha$ — некоторая константа. При этом условии основное соотношение (2) принимает вид $$N = (c_{P}q +\alpha)\Delta T \;(7)$$
		Следовательно, при фиксированном расходе воздуха ($q = const$ ) подводимая мощность и разность температур связаны прямой пропорциональностью($\Delta T(N)$ — линейная функция).
		
	\subsection{Ход работы}
	\begin{enumerate}
		\item Подготовим к работе газовый счетчик: проверим, что он заполнен  водой, установим счетчик по уровню.
		\item Охладим калориметр до комнатной температуры.
		\item Включим вольтметр, предназначенный для измерения ЭДС термопары. 
		\item Запишем показания компнатной температуры и давления. $$T_{0} = 296.25 \; ^\circ C, P_{0} = 100000 \pm 13 \; {Па} $$
		\item С помощью газового счетчика и секундомера измерим максимальный расход воздуха $\frac{\Delta V}{\Delta T}$ (в л/с). Измерения представлены в таблице 1. По найденным значениям определим среднее значение расхода и массовый расход воздуха $q_{max}$ [г/с].
		
		$$q = \rho_0 \frac{\Delta V}{\Delta t} = \frac{\mu P_0}{RT_0} \frac{\Delta V}{\Delta t}.$$
		
		Относительная погрешность косвенных измерений может быть найдена по формуле $$\varepsilon_{q} = \sqrt{(\frac{\sigma_{T_0}}{T_{0}})^2+(\frac{\sigma_{P_0}}{P_{0}})^2+ (\frac{\sigma_t}{t})^2}$$ 


	\begin{table}
	\begin{center}
	\begin{tabular}{|c|c|c|c|}
				\hline
				$\Delta V, л$ & $\Delta t, c$ & $\frac{\Delta V}{\Delta t},\frac{л}{с}$ & $q_{1},\frac{г}{c} $ \\
				\hline
				5 & 29.83 & 0.16762 & 0.195279 \\ 
				\hline
				5 & 30.05 & 0.16639 & 0.194842  \\
				\hline
				5 & 30.01 &  0.16661 & 0.195102  \\
				\hline
				5 & 30.07 & 0.16628 & 0.194712 \\
				\hline
				5 & 30.07 & 0.16628 & 0.194712 \\
				\hline
	
	\end{tabular}
	\end{center}
	\caption{Измерение расхода воздуха}
	\end{table}
	$$\overline{\frac{\Delta V}{\Delta t}} = 0.166636 \; \frac{л}{с} ,\; \overline{q_{1ср}} = 0.1949294 \; \frac{г}{с} $$
	
	Погрешность массового расхода может быть найдена по формуле: $$\sigma_{q_{1ср}сл} =  \sqrt{\frac{\sum_{i=1}^{7} (q_{1cp,i}-\overline{q_{1ср}})^2}{6}} = 0.0002 \; \frac{г}{c}.$$
	
	Окончательное значение: $$q_{1cp} = 0.1949 \pm 0.0002 \; \frac{г}{с} $$
	
	\item Оценим величину тока нагревателя $I_{0}$, требуемого для нагрева воздуха на $\delta T = 1 {К}$.
	
	Определим теоретическое значение удельной теплоемкости воздуха при постоянном давлении $C_{теорp} \; \frac{Дж}{г\cdot K}$, считая воздух смесью двухатомных идеальных газов: $Cp = 3.5R\mu \approx 1 \; \frac{Дж}{г\cdot K}.$
 	
 	Оценим минимальную мощность $N_0$, необходимую для нагрева газа при максимальном расходе. $N_{0} = c_{p}q_{1}\Delta T \approx 0.195 {Вт}.$
	
	Учитывая, что сопротивление проволоки нагревателя составляет приблизительно $R_{н} \approx 35 {Ом}$ и в процессе опыта практически не меняется, искомое значение тока $I_{0} = q N_{0} R_{н} \approx 0.075 \; {А}.$
	
	\item Проведем измерение зависимости разности температур от мощности нагрева $\Delta T(N)$ при максимальном расходе воздуха $q_0 = q_{1cp}.$
	\begin{table}

	\begin{center}
	\begin{tabular}{|c|c|c|c|c|c|c|c|c|}
		\hline
		$I, мA$ & $U, B$ & $N, Вт$ & $R_н, Ом$ & $\varepsilon, \muВ$ & $ \Delta T, K$
		\\
		\hline
		107.35 & 3.780 & 0.40578 & 35.2119 & 74 & 1.81818
		\\
		\hline
		152.48 & 5.370 & 0.81882 & 35.2177 & 149 & 3.66093
		\\
		\hline
		170.71 & 6.013 & 1.02648 & 35.2235 & 187 & 4.59459 
		\\
		\hline
		187.72 & 6.612 & 1.24120 & 35.2227 & 225 & 5.52826 
		\\
		\hline
		199.80 & 7.038 & 1.40619 & 35.2252 &  256 & 6.28993
		\\
		\hline
	\end{tabular}
	\end{center}
	\caption{Измерение $\Delta T (N) \; {при} \; q_{1ср}$}
	\end{table}
	Следует отметить, что погрешность измерения тока: $\sigma_{I} = 0.01 \; мA$, а  напряжения: $\sigma_{U}= 0.01 \; В$, $\sigma_{\varepsilon}= 1 \; \muВ$

	Завершив первую серию измерении, охладим калориметр до комнатнои температуры.

 Данные представлены в таблице 2.
 
	Для этого отключим источник питания нагревателя, откроем кран К и продуем калориметр при максимальном расходе воздуха до тех пор, пока показания ЭДС не достигнут нуля.
	
	
	\begin{table}
	\begin{center}
	\begin{tabular}{lr}
	\begin{tabular}{|c|c|c|c|}
	\hline
	$\Delta V, л$ & $\Delta t, c$  & $\frac{\Delta V}{\Delta t},\frac{л}{с}$ & $q_1, \frac{г}{c}$\\
	\hline
	5 & 68.85 & 0.07262 & 0.085040 \\
	\hline
	5 & 68.46 &  0.07304 & 0.085524 \\
	\hline
	5 & 68.78 & 0.07270 & 0.085126 \\
	\hline
	5 & 68.88 & 0.07259 & 0.085003 \\
	\hline
	5 & 68.96 & 0.07251 & 0.084904 \\
	\hline
	\end{tabular}
	
	
\end{tabular}
\end{center}
\caption{Измерения второго расхода $q_{2}$ }
\end{table}
	Проведем аналогичные измерения для других значений расхода воздуха.
	Данные представлены в таблице 3 и 4. Погрешности рассчитаны аналогично $q_{1cp}.$

\begin{table}
	\begin{center}
		\begin{tabular}{lr}
			\begin{tabular}{|c|c|c|c|c|c|}
				\hline
				$I, мA$ & $U, B$ & $N, Вт$ & $R_н, Ом$ & $\varepsilon, \muВ$ & $ \Delta T, K$\\
				\hline
				77.79 & 2.743 & 0.21338 & 35.2616 & 77 & 1.89189 \\
				\hline
				110.17 & 3.882 & 0.42768 & 35.2365 & 160 & 3.93120 \\
				\hline
				131.11 & 4.620 & 0.60573 & 35.2736 & 227 & 5.57740 \\
				\hline
				149.92 & 5.282 & 0.60573 & 35.2321 & 300 & 7.37101 \\
				\hline
				161.56 & 5.778 & 0.93349 & 35.2638 & 353 & 8.67322 \\
				\hline	
			
			\end{tabular}
		\end{tabular}
	\end{center}
	\caption{Измерение $\Delta T (N) \; {при} \;  q_{2}$}
\end{table}

		Погрешности будем считать по слудующим формулам: $$ \sigma_{\Delta T} = \Delta T \frac{\sigma_{\varepsilon}}{\varepsilon}, \sigma_{N}= N\sqrt{( \frac{\sigma_{I}}{I})^2 + (\frac{\sigma_{U}}{U})^2} \approx N \frac{\sigma_{U}}{U}$$
	
		\begin{table}
		\begin{tabular}{lr}
		\begin{tabular}{|l|l|l|l|l|l|l|l|l|l|l|}
			\hline
			q, $\frac{г}{с}$ & \multicolumn{4}{|c|}{0.1949} & \multicolumn{4}{|c|}{0.0851}
			\\
			\hline
			k &  $\Delta T, \; ^\circ C$& $\sigma_{\Delta T}\; ^\circ C$ & $N$, Вт & $\sigma_{N}$, Вт & $\Delta T, \; ^\circ C$& $\sigma_{\Delta T}\; ^\circ C$ & $N$, Вт & $\sigma_{N}$, Вт
			\\
			\hline
			1 & 1.818 & 0.006 & 0.4057 & 0.0001 & 1.892 &  0.003 & 0.2133 & 0.0001
			\\
			\hline
			2 & 3.660 & 0.006 & 0.8188 & 0.0001 & 3.931 & 0.003 & 0.4276 & 0.0001
			\\
			\hline
			3 & 4.594 & 0.006 & 1.0264 & 0.0001 & 5.577 & 0.003 & 0.6057 & 0.0001
			\\
			\hline
			4 & 5.528 & 0.006 & 1.2412 & 0.0002 & 7.371 & 0.003 & 0.7918 & 0.0002
			\\
			\hline
    			5 & 6.289 & 0.006 & 1.4062 & 0.0002 & 8.673 & 0.003 & 0.9335 & 0.0002
			\\
			\hline
		\end{tabular}

		\\

		



		\end{tabular}
	\caption{ погрешности, $\Delta T \; и \; N$}
	\end{table}


	После завершения опытов выключим источник питания нагревателя и мультиметры. Кран К откроем для максимального продува воздуха через калориметр.
	
\item Построим на одном графике зависимости $\Delta T (N)$ при разных значениях $q$.
\begin{figure}[H]
\center
\includegraphics[scale=1]{TN.png}
\caption{График зависимости $\Delta T(N)$}
\end{figure}
Полученные зависиомсти из графиков:
\[
y_1 =k_1 x_1 + b_1;\qquad y_2 = k_2 x_2 + b_2;
\]
\[
k_1 = 4.4598 \pm 0.0002; \quad b_ 1 = (8.68 \pm 0.07)\cdot 10^{-3}; 
\]
\[ 
k_2 = 9.4236 \pm 0.0011; \quad b_ 2 = -0.1124 \pm 0.0003;
\]
Найдем $\alpha$ и $c_P$, решив систему уравнений:
\[
	\left\{
		\begin{aligned}
			& c_P\, q_1 + \alpha = \frac{1}{k_1} \\
			& c_P\, q_2 + \alpha = \frac{1}{k_2}
		\end{aligned}
	\right.
\]
Путем математических преобразований получаем:
\[
\begin{aligned}
	 c_P = \frac{k_2 - k_1}{(q_1 - q_2)\, k_1\, k_2}; & \qquad  \alpha = \frac{k_2-k_1-c_P(q_1+q_2)}{2\,k_1\,k_2}. \\
	 c_P = 1076\ \frac{\text{Дж}}{\text{кг К}}; & \qquad  \alpha = 0.055\ \frac{\text{Дж}}{K} 
\end{aligned}
\]
Оценим погрешности:
\[
	\sigma_{c_P} = c_P \sqrt{\left(\frac{\sigma_{k_1}}{k_1}\right)^2 + \left(\frac{\sigma_{k_2}}{k_2}\right)^2} \approx 14
\]

\[
			\begin{aligned}
			& \fbox{$ c_P  =   1.076 \pm 0.014 \frac{\text{Дж}}{\text{г К}}$} \\
			& \fbox{$ \alpha = 0.055\pm 0.001\ \frac{\text{Дж}}{K} $}
			\end{aligned}
\]
\section*{Вывод}
Найденное значение молярной темлоёмкости $c_P = 1076 \pm 14 \frac{\text{Дж}}{\text{кг К}}$ чуть выше табличного, равного 1030 $\frac{\text{Дж}}{\text{кг К}}$. Это может быть вызвано погрешностью определения времени при измерении расхода воздуха.

\end{enumerate}

\end{document}
